\clearpage
\addcontentsline{toc}{chapter}{Abstract}

\begin{abstract}

One of the most essential problems which winegrowers are facing today is the inability for gaining a unified and robust multi-parametric management system, which can collect simultaneously imperative information for evaluating vineyard yield density, sequencing when and where specific cells should be diluted, monitoring irrigations, or even preventing damages by invasive wild bar species. Nevertheless, as precision technology has driven the farming revolution of recent years, precise yield surveying and monitoring from above will drive the following one. UAS can capture highly accurate images of a vineyard at a far greater resolution than satellite imagery provides, even under cloudy conditions, covering huge area in a single flight without the cost and burden of manned services. Here, we suggests to exploit UAS technology, mounted by highly sensitive optical, acoustic and radio frequencies sensors to form an automated unified management platform, which would continuously monitor all aspects related to proper vineyard functionality such as: visual inspection, elevation modeling, soil property \& moisture analysis, yield and foliage stress analysis, water management, erosion analysis, yield and foliage counting, irrigation scheduling and maturity evaluation. We will use image processing algorithms, developed at the R\&D center, to calibrate pixel values into practical vegetation indices such as Leaf Area Index (LAI) or Normalized Difference Vegetation Index (NDVI) to create a reliable reflectance maps of the yield and foliage vineyard abundance. This will emphasize exactly which sections inside the vineyard need closer examination, meaning less time spent scouting and more time treating the plants that actually need it. Ultimately, this will lead to boosting yields and reducing the vineyard management costs. 

\end{abstract}